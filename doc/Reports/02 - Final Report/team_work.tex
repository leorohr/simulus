\section{Team Work}
\label{sec:team_work}
Critical to the success of our project has been our ability to work as a collective with a clear objective. Teamwork translates into good practise by efficient organisation and harmony. In this section we describe the way in which the team collaborated and the strategies formed to overcome the challenges expected in developing software as a group.

\subsection{Development Process}\label{ss:dev-process}
As outlined in the initial report, we selected to follow an Iterative and Incremental Development (IID) process. This was evident in the group's Gantt chart\footnotemark[01] having planned short development cycles. These cycles were comprised of a requirements and planning task to identify what was to be achieved and how it was to be accomplished. We then progressed to possible design solutions that would meet the requirements. We developed the feature and then carried out testing against it. Collectively evaluating the feature and our efforts in that week was the final stage of a cycle. Each feature had an exit point, where once the requirements where met, we could progress. In order to establish the priority for each feature and thus the order in which it would be implemented, the perceived value gained by each was our indicator solely based on the MoSCoW analysis, carried out in subsection~\ref{ss:req-editor-enum}.
\footnotetext[01]{We refer the reader to the team's GitHub repository for a copy.}

A notable benefit of the above approach was that it encouraged modular design, facilitating decomposition of large, undefined requirements into manageable features: "Map drawing feature" was refined to "land tiles have three different instances; grass, dirt and water" and "simulator has export button" to "ability to export simulation parameters from within the simulator", etc.  
Probably the most beneficial aspect was that we were able to ascertain our development progress early on and thus adjust the time dedicated to the project, manoeuvring group members where additional effort was required. The iterative feedback not only allowed us to evaluate the software we were developing but our own techniques and group dynamics on a weekly basis.

\subsection{Team Policy}
The team's identified approach to management was wholly democratic opting to vote on most points of contention, from the triviality of tile colours to the more pressing decision of using a continuous model over the discrete prototype.

\begin{lstlisting}[caption={Decision Making Agreement}]
1 Each individual has the opportunity to voice their concerns
2 Each individual has an equal vote on an issue
3 The majority vote is the deciding vote
\end{lstlisting}
 
Although one may have concern that this policy could lead to inaction, we found that system worked well for us on this occasion. Perhaps this was in part due to the individual members, their personalities and working mentality. We had minimal, if any, conflicts to resolve, as a result of open discussion and brainstorming within the allocated weekly meetings\footnotemark[02] and Facebook group.
\footnotetext[02]{We refer the reader to the team's GitHub repository for minutes.}

\subsection{GitHub}
Our tool set for collaboration was minimal yet sufficient listing Facebook, WhatsApp and GitHub. We fully embraced GitHub and most of it's available features. We have found GitHub to be an accomplished tool for software projects and can fully appreciate why it is popular in industry and open-source projects. For several of us it was our first exposure to distributed version control and source code management but we can be confident that all facets of our group work was enhanced by the use of this tool.

The team's usage of GitHub was immediate but was largely adopted once we had made the decision to change to a continuous model. Whereas previously we had worked in a single branch, we created an additional two - a 'model' and 'editor' for each area of the project.  This allowed the two development sub teams to work independently. At strategic points and at the end of a development cycle (subsection~\ref{ss:dev-process}), we would aim to merge the two branches with the necessary members present. As useful was the 'issues' feature where any individual may create a task for himself or for the team as a whole. We were able to translate functional requirements, to almost atomic levels in some cases, and assign them to team members. The ability to post questions and comments against an issue was helpful and the tagging feature was a convenient method to quickly prioritise tasks. The team  did attempt to create suitable Wiki pages for the project but this was not entirely accomplished as expanded on in the next section.

\subsection{Challenges}
Inevitably we did face some challenges, none that were detrimental to the overall completion and achievement of our aims. However the basis of learning from our experience is that we are clear on what these were and how they were overcome. 

Having praised the value of using a Git repository above, it is important to understand that good communication is still paramount when working through such a system. Not being aware of the current classes a colleague was working on led to us suffering a delay when a commit was made without reviewing the conflicts. Increased communication through private messaging and the GitHub issues feature, meant this was no longer an issue once identified. All members made others aware of their current activity, progress and subsequent tasks to be tackled.

The team felt additional informational pages should have been posted to our Git's Wiki. An effort was made initially but not sufficiently to fully document the project. Unfortunately we were unable to find the time to make this a reality but it would be feasible to complete at a later date.

