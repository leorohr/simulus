\section{Introduction}
With the development and growth of cities comes an increase in traffic volume and congestion. Traffic congestion can be caused by many factors, some decrease the amount of road availability such as, accidents, breakdowns or road works, while others increase the volume of traffic, for example rush hour, school-runs, sporting events and holiday traffic. The effects of congestion cost the economy billions of pounds each year in delays and wasted time \cite{arnott1994economics,FinancialTimes} with these costs expected to rise year on year. In order to ease congestion, traffic management policies can be put in place, for example reduced speed limits, traffic lights, low carbon emission restrictions, bus or high-occupancy vehicle lanes, or tolls such as the London congestion charge. While these methods do have an effect on the volume of traffic \cite{TFLImpacts} new road infrastructure is often required in order to reduce wasted drive time.
The construction of new roads is extremely expensive \cite{BBC} and as such it is important to ensure that any new road systems, or changes to existing infrastructure, will have the desired impact on congestion. Traffic simulation software enables the modelling of road infrastructure and traffic, along with the analysis of varying levels of traffic volume and the effects of different traffic management policies on the modelled infrastructure.

\paragraph{}
Here we investigate the construction of traffic simulation software over a short period of time by a small team of five developers.
The resulting software system is comprised of a simulator and an editor. The editor part of the system allows a user to design road infrastructure, which can then be executed in the simulator. Roads are made up of four lanes with two lanes travelling in either direction, traffic light controlled intersections can be inserted to model crossroads and T-junctions, and road works can be added to close lanes or sections of road.

\paragraph{}
The simulator is a microscopic discrete time model with continuous state and space that models the actions of individual vehicles. Drivers have several different behaviours: 'cautious', 'normal', or 'reckless' and the behaviour of a driver may change during their journey. The volume of traffic, the speed limit, the ratio of cars to trucks and the recklessness of drivers, as well as the overall speed that the simulator runs at, are all adjustable. Emergency vehicles can also be spawned into the simulation. Various statistics, including average speed and queue time, are recorded during simulation. These statistics are displayed to the user in real-time in the form of graphs and can also be saved and output in numerical form for in-depth analysis.