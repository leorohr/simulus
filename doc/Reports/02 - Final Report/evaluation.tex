\section{Evaluation \& Conclusions}
\label{ss:eval_sim}
In terms of the realisation of the initially set requirements, we were able to meet them all except for the inclusion of external map sources (requirement 3b in section~\ref{sec:reqs}). This, however, was associated with the "could have" priority class and hence is an acceptable shortcoming. As our map model is based on fixed sized tiles, it would require intense translation to reproduce a map from sources like Google Maps or KML files due to their high resolution. 

\subsection*{Design}
At the beginning of the project, we implemented a model-view-controller pattern. However, during the course of the implementation it became apparent that strictly following this pattern is impractical for accurately modelling an instance of the real world. In the current state of the software, the initially separated model and view are combined in the view package. An example of this consolidation are the vehicle classes. They extend the JavaFX class \textit{Rectangle}, which is used to visualise the different types of vehicles. Nonetheless, vehicles also know their position in the map model and compute their movement on their own - both functionality that belongs to the model in a strict MVC architecture.

The methods that compute vehicle movements require numerous checks of a vehicle's environment. An example would be the method by which a car overtakes another. Multiple tiles of the adjacent lane must first be checked for the conditions to be met for this manoeuvre to take place. The exact tiles that need to be checked are currently dependant upon the direction of the vehicles' movement and hence these checks come with a lot of nested conditional statements. The \textit{attemptOvertake} method in the \textit{Vehicle} class exhibits a cyclomatic complexity of 34, being the most complex method in our source code (see fig.~\ref{fig:heatmap}). Reducing the complexity for this method along with others would most likely improve the scaling of the simulator's performance with a higher number of vehicles. Implementing these changes would require a major redesign of the simulation logic and was hence deemed as infeasible to do within the short duration of this project.

Maps are persisted using XML. Each map-file contains data for every single tile on the map. Unfortunately, the XML specification we used does not specify the grouping of tiles at an intersection, forcing us to recreate the intersections as soon as an intersection tile is encountered while reading the map. This means that the algorithm will create an intersection at the encountered tile and hence create the tiles 3 columns to the right and 3 rows down without reading any further tile information from the XML file relating to that intersection. To ensure that intersections are created correctly, we have to create a 2D boolean array the size of the map (i.e. 40x40, 60x60 or 80x80) to keep track of tiles that have been already successfully read.

\subsection*{Model}
As described in section \ref{subsec:design}, we began using a discrete space model and changed to a continuous model later on. A discrete space model, often referred to as cell automaton, is easy and quick to implement because the movement of vehicles through the grid can be simulated by simply storing a two-dimensional array of cells and moving each car on a central clock tick. Nevertheless, the discrete approach did not allow us to introduce vehicle-specific attributes like acceleration and different movements speeds.
 This change was a product of improved thinking, as we realised that the restrictions imposed by a discrete space representation outweigh the benefits of the easier implementation that comes with it.

Our model currently only supports roads with four lanes and dual carriageways. Adding support for different lane models and road sizes will increase the capabilities and functionality of both simulator and editor; we consider this a suitable task for further development of the software. In addition, future work could involve the evolution of the current time model from discrete to continuous. A wholly continuous approach is realistically the most accurate depiction of the real world. 

\subsection*{Team Work}
All five group members were involved in the creation of the code. Collaboration and communication was eased through extensive use of the Git Issues feature to track problems and progress in solving them (cf. section~\ref{sec:team_work}). Working in separate branches of the repository when working on the same files or related behaviour allowed every team member to contribute to the codebase continuously.  

\subsection*{Conclusion}
Implementing a sophisticated traffic simulation that allows for accurate representation of real-world phenomena is a well developed research topic in computer science; this research exposing a multitude of different approaches and models. In this project we developed a traffic simulation that allows for microscopic vehicle behaviours, dynamic simulation governance and user-defined road networks. 

The produced software is split into two standalone modules: a simulator and a map editor. The maps built in the editor can be stored and then loaded by the simulator. The simulation is a combination of a cellular automaton and a continuous space model. Vehicle's move through the road network independently of the underlying cell model, but while doing so, also occupy all the cells that they currently intersect. This eased the implementation of collision avoidance mechanisms while keeping the vehicle's movements flexible. The evaluation of the traffic network is triggered by a central clock. That means we examine the simulation in time-slices, above referred to as tickrate. Each time-slice is one step in the simulation. This combination of cell-automata and the continuous space model makes our approach a hybrid between these two.

