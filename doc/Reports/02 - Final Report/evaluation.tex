\section{Evaluation}
\subsection{Simulator}\label{ss:eval_sim}
In terms of the realisation of initially set requirements, we were able to meet them all except for the inclusion of external map sources (requirement 3b in section~\ref{sec:reqs}). This, however, was associated with the "could have" priority class and hence is an acceptable shortcoming. As our map model is based on fixed sized tiles, it would require intense translation to reproduce a map from sources like Google Maps or KML files due to their high resolution. 

As described earlier we started off with a discrete space model and changed to a continuous model later on. This change was a product of improved thinking, as we realised that the restrictions imposed by a discrete space representation outweigh the benefits of the easier implementation it comes with.

At the beginning of the project, we implemented a model-view-controller pattern. However, during the course of the implementation, it became apparent that strictly following this pattern is impractical for this project. In the current state of the software, the initially separated model and view are now combined in the view package. An example for this consolidation are the vehicle classes. They extent the JavaFX class \textit{Rectangle}, which is used to visualise the different types of vehicles. Nonetheless, vehicles do also know their position in the map model and compute their movement on their own - both functionality that belongs to the model in a strict MVC architecture.

The methods that compute vehicle movements require a lot of checks for the vehicle's environment. For example, to overtake another car the adjacent lane and the space in front of the slower car has to be checked. Which tiles have to be checked currently depend on the direction of the vehicles' movement and hence these checks come with a lot of nested conditional statements. The \textit{attemptOvertake} method in the \textit{Vehicle} class exhibits a cyclomatic complexity of 34, therewith being the most complex method in our source code (see fig.~\ref{fig:heatmap}). Reducing this and other methods' complexity would most likely improve the performance of the simulation with a higher number of vehicles. Reducing this complexity would require a major redesign of the simulation logic and was hence deemed as infeasible to do within the short duration of this project.

All five group members were involved in the creation of the code. Collaboration and communication was eased through extensive use of the Git Issues feature to track problems and progress in solving them (cf. section~\ref{sec:team_work}). Working in separate branches of the repository when working on the same files or related behaviour allowed every team member to contribute to the codebase continuously.  

\bigskip
* recreation of intersections on mapLoad()
* only fourlane roads
* no curves, roundabouts
