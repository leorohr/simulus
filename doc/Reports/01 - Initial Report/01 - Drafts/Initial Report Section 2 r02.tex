\documentclass[11pt,a4paper]{article}
\usepackage[utf8]{inputenc}
\usepackage{amsmath}
\usepackage{amsfonts}
\usepackage{amssymb}
\usepackage{graphicx}
\usepackage{fancyhdr}
\pagestyle{fancy}
\rhead{Team Simulus}
\rfoot{r02}
\begin{document}


\part{Planning \& Organisation}

\section{Team Working Policy}
\paragraph{}
Through our formal discussions in project meetings, we have established very early on that we will be adopting a cooperative, mutually-beneficial and largely democratic approach to management.  It is felt that all group members are equally determined to attain maximum results from our efforts and believed our equal abilities and proposed effort would lead to an equal input on project decisions.  One member, was nominated to be the Project Coordinator for formal submissions and as a point of contact.

\section{Development Methodology}
\paragraph{}
Taking into consideration the differing methodologies to software development and in particular the challenges we presumed to face, we have chosen to follow the ‘Iterative and Incremental’ development approach.  This technique can be summarised as “gradual increase in feature additions and a cyclical release and upgrade pattern.  Iterative and incremental software development begins with planning, continues through iterative development cycles involving continuous user feedback and the incremental addition of features concluding with the deployment of completed software at the end of each cycle.“\footnote{http://www.techopedia.com/definition/25895/iterative-and-incremental-development}

\begin{figure}[hbtp]
\centering
\includegraphics[scale=0.4]{iterative-model.jpg}
\caption{The ‘Iterative \& Incremental' Development Approach}
\end{figure}

We feel this working pattern is most suitable making use of weekly feedback sessions and incrementally developing on top of the base functionality defined.  This in conjunction with the tight schedule, team members’ workload and simple process workflow make this development approach the best choice in our opinion.

\section{Project Roles}
\paragraph{}
Having researched the problem area and identified the main phases of this project, then further divided each phase into tasks, we agreed the following role definitions:

\begin{enumerate}
  \item \textbf{Research \& Designer} - responsible for delving into the literature, learning from existing methodologies and suggesting solutions to our challenges.
  \item \textbf{Developer (Graphics \& GUI)} - responsible for translating functional graphical requirements, with the aid of Designers, into deliverables.
  \item \textbf{Developer (Core Functionality)} - responsible for translating functional requirements of the model and backend into deliverables.
  \item \textbf{Documentation Analyst} - responsible for documenting the system implementable, the current status of the project and report writing.
  \item \textbf{Project Coordinator} - responsible for the planning and coordination of team members for the currently selected phase of work and other operational issues i.e. management of GitHub, wiki and other workspaces.
\end{enumerate}

It is paramount to note that we are not ultimately rigid in our role selection.  Individuals may take on the responsibility, possibly more than one, of each of the defined roles for a set period.  This allows all team members to experience the challenges faced from all perspectives and has the added advantage of flexibility i.e. to re-assign an individual should they be required elsewhere.

\section{Collaboration}
\paragraph{}
All project members carrying out development work are mandated to deposit their code using Github’s functionality.  Furthermore those producing research, design or planning output must also upload to the designated GitHub repository.  We have already mapped our project plan into issuable tasks that are being assigned to individual(s) to work on and progress.  In addition GitHub has an excellent wiki feature where contributors can document the many facets of the project.  For example, a wiki page to outline and define the interfaces for the different coding branches e.g. GUI and core code.
\paragraph{}
Although the state of collaborative platforms are feature-rich, we strongly feel there is no replacement for face-to-face discussion.  Thus we have scheduled as a minimum one major and minor project meeting a week.  The aim of these meetings are to discuss current progress, demonstrate working functionality of the traffic simulator and agree a roadmap for the following week.
\paragraph{}
We are also effectively employing a Facebook group for open discussion and operational issues like absence.  Evernote, a multi-platform collaboration system, is being employed for meeting minutes and hosting of temporary or large files.  Finally the group has found the use of instant messaging application WhatsApp to be very beneficial with time sensitive messages or group notifications.

\section{Peer Assessment}
\paragraph{}
As a direct result of our agreed working policy and ethos, encouraging a proactive attitude, we have opted to award all team members equal weighting of the final result.  At the point prior to submission we will meet to ratify equal contribution and if there are any concerns, a majority vote will be issued.  For example, should the majority of members feel one or more individuals contributed less than or greater than the average, then a vote would be taken for the deduction/awarding of a set number of points.  At this stage if there is agreement then this is submitted as approved allocation.  It was unanimously decided at the project start that any decision on peer assessment would be dealt with internally and only as a last resort should "failure to agree distribution" be submitted.

\end{document}