\documentclass[11pt,a4paper]{article}
\usepackage[utf8]{inputenc}
\usepackage{amsmath}
\usepackage{amsfonts}
\usepackage{amssymb}
\usepackage{graphicx}
\usepackage{fancyhdr}
\pagestyle{fancy}
\rhead{Team Simulus}
\rfoot{r01}

\begin{document}

\part{Project Description}

\section{Introduction}
\paragraph{}
The problems related to traffic and transportation intensify greatly as towns and cities develop and grow. The field of Traffic Engineering aims to plan transportation with reasonable precision for the sake of avoiding costly, and often irreversible, mistakes. Through an application of basic knowledge and theories in the field, road systems can be developed effectively where a trial and error approach is simply impractical.

\paragraph{}
Only by systematic evaluation can we upgrade our systems and discard ineffective designs. A direct approach to evaluating the design possibilities in transportation systems is by the use of computer software that simulates an instance of the real world. Traffic Simulation is the art of mathematically modelling such systems using software to help better the planning, design and management of them.

\section{Aims}
\paragraph{}
The aim of this project is to develop Traffic Simulation software as a team that can effectively fulfil these goals. We will be researching, planning, designing and implementing a software product capable of simulating traffic whilst paying close attention to group collaboration and work ethic.

\paragraph{}
We developed a hierarchy of aims that highlight the level of importance of each goal by using MoSCoW analysis. Naturally, aims with a higher level of importance should be wholly completed before work towards the other goals has begun. In this way, the chance of producing a complete and well-functioning software artefact is maximised. The produced aims are listed below by order of importance:

\subsubsection{Must}
\begin{enumerate}
  \item The system must simulate individual vehicles operating in different parts of a road network.
  \item The system must deal with vehicle entrance and exit points in a way that simulates a real world environment.
  \item The system must allow for the individual behaviour of vehicles.
\end{enumerate}

\subsubsection{Should}
\begin{enumerate}
  \item The system must allow different traffic management policies to be plugged in and compared.
  \item The system should have support for emergency services.
  \item The system should allow the user to create the map in which to simulate traffic.

\end{enumerate}

\subsubsection{Could}
\begin{enumerate}
  \item The system should be able to save maps to a file.
  \item The system should be able to load maps from a file.
  \item The system could use information from Google Maps to generate the area in which to simulate traffic.
\end{enumerate}

\subsubsection{Wont}
\begin{enumerate}
  \item The system will not be rely on any existing traffic simulation platforms, engines or modelling frameworks.  We restrict our efforts to the use of APIs of our development language and toolset.
\end{enumerate}

\section{Progress}
\paragraph{}

\end{document}