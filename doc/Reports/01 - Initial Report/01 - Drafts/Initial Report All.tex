\author{Team Simulus}
\title{7CCSMGPR Initial Report}
\date{06/02/2015}
\documentclass[11pt,a4paper]{article}
\usepackage[utf8]{inputenc}
\usepackage{amsmath}
\usepackage{amsfonts}
\usepackage{amssymb}
\usepackage{graphicx}
\usepackage{fancyhdr}
\pagestyle{fancy}
\fancyhf{}
\fancyhead[R]{Team Simulus}
\fancyfoot[R]{r05}

\begin{document}

\part{Project Description}

\section{Introduction}
\paragraph{}
The problems related to traffic and transportation intensify greatly as towns and cities develop and grow. The field of Traffic Engineering aims to plan transportation with reasonable precision for the sake of avoiding costly, and often irreversible, mistakes. Through an application of basic knowledge and theories in the field, road systems can be developed effectively where a trial and error approach is simply impractical.

\paragraph{}
Only by systematic evaluation can we upgrade our systems and discard ineffective designs. A direct approach to evaluating the design possibilities in transportation systems is by the use of computer software that simulates an instance of the real world. Traffic Simulation is the art of mathematically modelling such systems using software to help better the planning, design and management of them.

\section{Aims}
\paragraph{}
The aim of this project is to develop Traffic Simulation software as a team that can effectively fulfil these goals. We will be researching, planning, designing and implementing a software product capable of simulating traffic whilst paying close attention to group collaboration and work ethic.

\paragraph{}
We developed a hierarchy of aims that highlight the level of importance of each goal by using MoSCoW analysis. Naturally, aims with a higher level of importance should be wholly completed before work towards the other goals has begun. In this way, the chance of producing a complete and well-functioning software artefact is maximised. The produced aims are listed below by order of importance:

\subsection{Must}
\label{section:must}
\begin{enumerate}
  \item The system must simulate individual vehicles operating in different parts of a road network.
  \item The system must deal with vehicle entrance and exit points in a way that simulates a real world environment.
  \item The system must allow for the individual behaviour of vehicles.
  \item The system must compute statistics of interest such as Vehicle Miles Travelled (VMT), mean system speed, total system delay, etc. 
  \item The system must allow different traffic management policies to be plugged in and compared.
\end{enumerate}

\subsection{Should}
\begin{enumerate}
  \item The system should have support for emergency services.
  \item The system should allow the user to create the map in which to simulate traffic.
\end{enumerate}

\subsection{Could}
\begin{enumerate}
  \item The system could save and load maps from a file.
  \item The system could use information from Google Maps to generate the area in which to simulate traffic.
\end{enumerate}

\subsection{Won't}
\begin{enumerate}
  \item The system will not be rely on any existing traffic simulation platforms, engines or modelling frameworks.  We restrict our efforts to the use of APIs of our development language and toolset.
\end{enumerate}

\section{Progress}
\paragraph{}
Although the planning phase took longer than anticipated, ultimately it has been a fruitful exercise with very little deviation from the project milestones.  We have successfully selected a microscopic discrete simulation model for it's relative simplicity.  Using a Model-View-Controller(MVC) pattern a prototype has been developed with working visualiser.  At present we have met the first two requirements of \ref{section:must} by representing, in the model and visualiser, an intersection with two opposing roads each containing two lanes.  Vehicles are spawned from a designated entrance point once another leaves from the exist.    
\paragraph{}
The feedback from this discrete prototype has lead to us entertaining a continuous model approach.  Although the discreet method has the added advantage of being efficient for larger map representations, further research has shown we can achieve more realistic vehicle behaviour using a continuous model.

\pagebreak

\part{Project Organisation}

\section{Working Policy}
\paragraph{}
From the onset it was established we will be adopting a cooperative, mutually-beneficial and largely democratic approach to management.  Our determination to attain the maximum results from our proposed effort and ability, entitles each team member equal say.  An individual was nominated to be Project Coordinator thus acting as a point of contact, for formal submissions and for conflict resolution should it arise.

\section{Development Methodology}
\paragraph{}
Taking into consideration the particular challenges we presumed to encounter specifically time-management, a short development window and minimal exposure to working in teams, the ‘Iterative and Incremental’ development approach was favoured.  In general this technique involves the increase in feature additions and a cyclical release pattern with great emphasis on feedback.  The evaluation in iterative cycles allows for continued opportunity in planning, analysis \& design and implementation producing incremental addition of features.

\section{Project Roles}
\paragraph{}
Having identified the major phases of the project and critical tasks, we agreed the following role definitions:

\begin{enumerate}
  \item \textbf{Research \& Designer} - responsible for delving into the literature and suggesting solutions to our challenges.
  \item \textbf{Developer (Graphics \& GUI)} - responsible for translating functional graphical requirements, with the aid of Designers, into deliverables.
  \item \textbf{2 x Developer (Model Implementation)} - responsible for translating functional requirements of the model and back-end into deliverables.
  \item \textbf{Documentation Analyst} - responsible for documenting the system implementable, the current status of the project and report writing.
\end{enumerate}

It is paramount to note that we are not ultimately rigid in our role selection.  Individuals may take on the responsibility, possibly more than one, of each of the defined roles for a set period.  Team members are then able to experience the challenges faced from all perspectives and has the added advantage of flexibility i.e. to re-assign an individual should they be required elsewhere.

\section{Collaboration}
\paragraph{}
We have found GitHub to be exceptionally useful in managing collaborative effort, not only for the developers generating code but also those producing research, design or planning output.  At the start of each iteration cycle we are translating identified tasks from the project plan into issuable GitHub tasks that can be assigned to individual(s) to progress.  Furthermore GitHub has an excellent wiki feature where our contributors can document the many facets of the project.
\newline
In addition we are effectively employing a Facebook group for open discussion and operational issues such as absence.  Evernote, a multi-platform collaboration system, is being utilised for meeting minutes and hosting large files.  Finally the group has found the use of instant messaging application WhatsApp beneficial with time sensitive notifications.
\paragraph{}
Although the state of collaborative platforms are feature-rich we have found face-to-face meetings pivitol in our efforts.  Thus we have scheduled as a minimum one major and minor project meeting a week.

\section{Peer Assessment \& Conflict Resolution}
\paragraph{}
As a direct result of our working policy encouraging a proactive mindset, we have opted to award all team members equal weighting of the final result.  At the point prior to submission we will meet to ratify equal contribution and if there are any concerns, a majority vote will be issued.
This majority vote strategy is being used for wider issues of contention.  Conflict is resolved using the following rules:

\begin{enumerate}
  \item Each individual has the opportunity to voice their concern with the hope of preventing conflicts
  \item Each individual has an equal vote on an issue
  \item The majority vote is the deciding vote
\end{enumerate}

The largest point of contention is likely to be the completion of tasks or lack of by team members.  Therefore clearly defining roles and responsibilities from the onset and each individual knowing which deliverables are required is our identified method to ensure more favourable team dynamics.

\end{document}