\documentclass[11pt,a4paper]{article}
\usepackage[utf8]{inputenc}
\usepackage{amsmath}
\usepackage{amsfonts}
\usepackage{amssymb}
\usepackage{graphicx}
\usepackage{fancyhdr}
\pagestyle{fancy}
\rhead{Team Simulus}
\rfoot{r01}
\begin{document}


\part{Planning \& Organisation}

\section{Team Working Policy}
\paragraph{}
Through our formal discussions in project meetings, we have established very early on that we will be adopting a cooperative, mutually-beneficial and largely democratic approach to management unless it was found to be ineffective.  It was felt that all group members are equally motivated to attain the maximum results from our efforts and believed our equal abilities and proposed effort would lead to an equal input on project decisions.  One member, was nominated to be the project coordinator for formal submissions and as a point of contact.
\paragraph{}
We appreciate this approach maybe unorthodox.  However our aim is to learn from the challenges of working in a team and so to propose an alternative to project coordination whereby all members are self-organising and responsible.  Additionally due to project members’ commitments with other modules we felt this would allow us to collectively work towards the deadline, without the burden of organisation, planning and coordination on a single individual.  We add a caveat that should we notice the approach to be not conducive, a project manager would be elected.

\section{Project Roles}
\paragraph{}
Having researched the problem area and identified the main phases of this project, then further divided each phase into tasks, we agreed the following role definitions:

\begin{enumerate}
  \item \textbf{Research \& Designer} - responsible for delving into the literature, learning from existing methodologies and presenting this information, along with suggested solutions to our own challenges, to the project team.
  \item \textbf{Developer} – responsible for translating functional requirements, with the aid of Designers, into deliverables.
  \item \textbf{Documentation Analyst} – responsible for documenting the system implementable, the current status of the project and report writing.
  \item \textbf{Coordinator} – responsible for the planning and coordination of team members for the currently selected task or phase of work and other operational issues i.e. management of GitHub, wiki and other workspaces.
\end{enumerate}

We then went about identifying the key skill-sets in our talent pool and mapped them to the roles above, leaving the fifth individual as a ‘fluid’ work resource in each phase i.e. assisting another team member with tasks requiring additional effort.  It is paramount to note that we are not ultimately rigid in our role selection.  Individuals may take on the responsibility of each of the defined roles for a set period or as the projects requires and permits.  This allows all team members to experience the challenges faced from all perspectives and this would further gives the flexibility to re-assign an individual should they better suited elsewhere.

\section{Development Methodology}
\paragraph{}
Taking into consideration the differing methodologies to software development and in particular the challenges we presumed to face, we have chosen to follow the ‘Iterative and Incremental’ development cycle.  This approach can be summarised as “...a gradual increase in feature additions and a cyclical release and upgrade pattern.
Iterative and incremental software development begins with planning and continues through iterative development cycles involving continuous user feedback and the incremental addition of features concluding with the deployment of completed software package at the end of each cycle."\footnote{http://www.techopedia.com/definition/25895/iterative-and-incremental-development}


\begin{figure}[hbtp]
\centering
\includegraphics[scale=0.4]{iterative-model.jpg}
\caption{The ‘Iterative \& Incremental' Development Approach}
\end{figure}
\clearpage
\pagebreak

\section{Collaboration}
\paragraph{}
To ensure efficient communication such that all team members understand the current progress of the project, the required outcome of each project phase and the actionable tasks necessary, we have identified several suitable methods for collaboration.
\paragraph{}
Firstly, as required by the module specification, all team members have created a GitHub account and progressed through the introductory tutorial outlining the benefits of a version control systems.  All project members carrying out development work are required to deposit their code using Github’s functionality.  Furthermore those producing research, design or planning output must also upload to the designated GitHub repository.
\paragraph{}
We have already begun to map our project plan into issuable tasks in GitHub which can be assigned to individual(s) to work on and progress.  In addition GitHub has an excellent wiki feature where developers can provide an overview of the project, the aims and objectives, current progress as well as relevant material for project contributors.  For example, a wiki page to outline and define the interfaces for the different coding branches e.g. GUI and model functionality.
\paragraph{}
Although the state of collaborative platforms are feature-rich, we strongly feel there is no replacement for face-to-face discussion.  Thus we have scheduled, as a minimum, one project meeting a week from the project start on the 16th of January until the final meeting on the 20th of March.  These meetings are scheduled to start promptly for 14:00 every Friday.  The aim of these meetings is to discuss current progress, demonstrate working functionality of the traffic simulator and agree tasks for the coming week.
\paragraph{}
We are effectively employing a Facebook group for open discussion as well as operational issues such as absences.  Contributors are encouraged to post relevant research literature, feature suggestions and examples of existing implementation.  Evernote, a multi-platform collaboration tool, is being employed for recording minutes and hosting of temporary or large files.  Finally the group has found the use of instant messaging application WhatsApp to be very beneficial with time sensitive messages or group notifications.

\pagebreak

\section{Peer Assessment}
\paragraph{}
Unlike a conventional project where performance may be determined by an authoritative figure such a project manager or CIO/CTO, for this project we are encouraged to peer assess our effort and performance throughout the entirety of the project.
\paragraph{}
This requirement was a large influence in our choice of team members, motivation and working policy.  Recall that we are proposing a collaborative, mutually-beneficial and mostly self-organising project effort.  This was suggested as a direct consequence of opting to award all team members equal weighting of the final module result and thus encourage all individuals to commit maximum effort. 
\paragraph{}
At the point just prior to submission, we have agreed to ratify that all members have contributed equally and if there is any concern, a majority vote will be issued.  For example, should the majority of members feel one or more individuals contributed less or more than others, then a vote would be taken for the deduction/awarding of a set number of points.  At this stage, if there is agreement then this is submitted as a team allocation.  It was unanimously decided at the project start that any decision on peer assessment would be dealt with internally and only as a last resort should "failure to agree distribution" be submitted.  Please note this approach above is applicable so long as all team members continue to have equal voting and contribution.  If at any stage, a Project Manager was elected, he would have overruling decision of any majority vote.
\end{document}